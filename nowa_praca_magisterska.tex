\documentclass[a4paper,11pt,twoside]{article}
\usepackage[utf8x]{inputenc}
\usepackage{geometry}
\usepackage[T1]{fontenc}
\usepackage[polish]{babel}
\usepackage{graphicx}
\usepackage{amsmath}
%\usepackage{amssymb}
\usepackage{setspace}
\usepackage{fancyhdr}
\usepackage{wrapfig}
\usepackage{subfig}
\usepackage{hyperref}
\usepackage{sidecap}
\usepackage{theorem}
\usepackage{thc}
\usepackage{url}
\usepackage{booktabs}
\usepackage{multirow}
\usepackage{float}
\usepackage{wrapfig}
\usepackage{siunitx}
\usepackage{mathtools}
\usepackage{caption}
\usepackage{subfig}
\usepackage{tikz}
\setlength{\headheight}{15pt}

\geometry{left=3 cm,right=3 cm, top=3.5 cm, bottom=3.5 cm}
\fancyhf{}
\fancyhead[RO,LE]{\footnotesize{\leftmark}}
\fancyhead[LO,RE]{\thepage}

\pagestyle{fancy}

\renewcommand\maketitle{
\begin{titlepage}
\begin{center}

\textbf{ \begin{huge}
 Badanie i wdrożenie procedur niskopoziomowych korekcji przebiegu
sygnałów systemu odczytu danych Dragon dla teleskopów LST.
\end{huge}}

\vspace{0.3cm}

\begin{Large}
 \textit{Uniwersytet Łódzki, 2018}
\end{Large}

\vspace{0.3cm}


\begin{tabular}{ccc}
\LARGE{Pawel Gliwny} \vspace{0.15cm}\\
\end{tabular}


\vspace{0.7cm}
\large{Promotor\\dr. hab. Julian Sitarek}\\
\vspace{0.7cm}
\includegraphics[width=0.33\textwidth]{logo_wfis.png}\\
\vspace{0.7cm}
\begin{abstract}
LST (Large Size Telescope) jest największym z typów teleskopów czerenkowskich wchodzących w skład budowanego właśnie Obserwatorium CTA (Cherenkov Telescope Array). Konstrukcja pierwszego teleskopu LST zakończyła się 10 października. Poprzez obserwowanie słabych i ultrakrótkich błysków promieniowania czerenkowa, teleskopy czerenkowskie są w stanie obserwować elektromagnetyczne kaskady zainicjowane w atmosferze przez promieniowanie gamma. Światło czerenkowskie jest zbierane na kamerze i konwertowane na impulsy elektryczne korzystając z fotopowielaczy, sygnał ten jest później próbkowany z częstotliwością rzędu GHz. W teleskopach LST system odczytu danych Dragon oparty jest na chipie DRS4 (Domino Ring Sampler 4), który wymaga szeregu korekcji dla optymalnego działania. W pracy przedstawiam korekcje m.in. na linię bazową każdego próbkującego kondensatora, korekcje zależności od czasu ostatniego odczytu danego kondensatora, przewidywanie pozycji oraz interpolacje niegaussowskiego szumu. Korekcje zostały przetestowane na danych testowych zbieranych w laboratorium w Hiszpanii, oraz na pierwszych danych z prototypu LST. 
\end{abstract}

\Large{\today}

\end{center}
\end{titlepage}}

%\vfill







\begin{document}
\maketitle
\tableofcontents
\newpage
\setcounter{page}{1}


\section{Promieniowanie kosmiczne i astronomia gamma}
Astronomia jest nauką umożliwiającą badanie zjawisk fizycznych poza naszą planetą. Do początku XX wieku astronomia zajmowała się badaniem zjawisk termicznej emisji źródeł charakteryzujących się widmem Plancka. To się zmieniło wraz z odkryciem promieniowania kosmicznego (ang. \textsl{Cosmic Rays}) przez Victora Hessa w 1912 roku poprzez pomiar wzrostu gęstości liczbowej zjonizowanych cząsteczek wraz z wysokością\cite{particle_de_angelis}. W promieniowaniu kosmicznym obserwuje się cząstki o energiach sięgających $10^{20}$\,eV \cite{particle_de_angelis}, procesy termiczne nie są w stanie nadać cząstkom taką energię, więc muszą być one produkowane w procesach nietermicznych, które zazwyczaj charakteryzują się widmem potęgowym. \\
Promienie kosmiczne są cząstkami o największych znanych nam energiach, 
dzięki nim możemy badać przyspieszanie cząstek w najbardziej ekstremalnych warunkach. Ich badanie pomaga nam także lepiej rozumieć skład i ewolucje Wszechświata. Jednakże są to cząstki naładowane elektrycznie, więc są odchylane w polach magnetycznym podczas propagacji, co utrudnia nam określenie źródła ich powstania. Kierunek ich nadejścia wydaje się być (prawie) izotropowy. Możliwe jest tylko określenie kierunku przyjścia cząstek naładowanych promieniowania kosmicznego o najwyższych energiach, czym zajmuje się obserwatorium \textsl{Pierre Auger} w Argentynie \cite{auger_web} oraz obserwatorium \textsl{Telescope Array} na półkuli północnej \cite{telescope_array_web}. Aby, określić źródła musimy badać cząstki neutralne takie jak: neutrina i fotony o najwyższych energiach, czyli promieniowanie gamma. Neutrina z uwagi na bardzo mały przekrój czynny są trudne do wykrycia. Więc tematem mojej pracy jest astronomia gamma bardzo wysokich energiach $>30$\,GeV. W tym rozdziale przedstawiam najważniejsze informacje o astronomii gamma, mechanizmy powstania fotonów $\gamma$ oraz techniki detekcji.

\subsection{Astronomia gamma}
Promieniowanie gamma o bardzo wysokich energiach $\geq$ 30\,GeV ang.(\textsl{very high energy, VHE}) według dzisiejszej wiedzy jest wynikiem interakcji populacji relatywistycznych cząstek o bardzo wysokich energiach z otaczającą materią lub z niskoenergetycznymi polami promieniowania. Badanie fotonów gamma umożliwia nam poszerzanie naszej wiedzy na temat procesów przyspieszania cząstek o największych energiach, testowanie teorii fizycznych w ekstremalnych warunkach: bardzo silnej grawitacji (czarne dziury), bardzo silnego pola magnetycznego (gwiazdy neutronowe). Astronomia gamma dostarcza nam wyjątkowe narzędzia do badania wielu astrofizycznych zjawisk: pozagalaktyczne promieniowanie tła (\textsl{ang. Extragalactic background light (EBL)}), poszukiwanie ciemnej materii \cite{Gamma-ray_article}. Do tej pory zostało odkryte ponad 200 obiektów, które są źródłem fotonów o najwyższych energiach \cite{TeVCat}. Większość tych źródeł została odkryta za pomocą teleskopów czerenkowskich (ang. \textsl{Imaging Atmospheric Cherenkov Telescopes, IACT}), które zostaną opisane w tym rozdziale.\\
Początki astronomii gamma datuje się na lata 50. XX wieku.
Pierwszym znaczącym przełomem było odkrycie emisji w zakresie TeV z mgławicy Craba w 1989 roku przez teleskop Whipple \cite{whipple}.

Atmosfera ziemska absorbuje fotony gamma, więc bezpośrednim sposobem badania tych fotonów jest wysłanie misji kosmicznej. Jednakże strumień fotonów o najwyższych energiach jest bardzo mały, co wynika z mechanizmu produkcji tych fotonów. Na przykład mgławica Craba, jeden z najintensywniejszych obiektów będącym źródłem fotonów $\gamma$ wytwarza strumień jedynie $\sim$ 6 fotonów $\mathtt{m^{-2}}$ $\mathtt{rok^{-1}}$ na powierzchni ziemi o energii powyżej 1\,TeV \cite{IACT}. W celu badania kosmosu w tych zakresach energii potrzebna jest bardzo duża powierzchnia, na której można dokonywać detekcji, znacznie większa niż możliwa powierzchnia detektora w przestrzeni kosmicznej $\sim$ 1\,$\mathtt{m^{2}}$. Teleskopy czerenkowskie spełniają ten warunek poprzez detekcje promieniowania Czerenkowa powstającego poprzez oddziaływanie fotonu gamma z cząstkami atmosfery (głównie azotem). Ziemska atmosfera jest nieodłączną częścią procesu detekcji, spełnia rolę kalorymetru (czyli mierzy energię) oraz trackera (czyli mierzy kierunek przyjścia cząstki). \\
\subsection{Mechanizmy produkcji promieniowania Gamma}
Źródłami wysokoenergetycznego promieniowania kosmicznego oraz promieniowania gamma są pozostałości po supernowych, szybko rotujące obiekty takie jak pulsary i gwiazdy neutronowe, aktywne jądra galaktyk, układy podwójne ze zwartym obiektem (gwiazdą neutronową lub czarną dziurą). W tych obiektach promieniowanie gamma jest produkowane przez różne mechanizmy \cite{astro_particle}:
\begin{itemize}
\item {\bf{Promieniowanie synchrotronowe}}
produkowane jest przez relatywistyczne naładowane cząstki (głównie elektrony) w polu magnetycznym w którym są hamowane, w wyniku czego produkują nietermiczne promieniowanie elektromagnetyczne. Moc promieniowania $P$ zależy do energii cząstki $E$ oraz natężenia pola magnetycznego $B$ w postaci $P \sim E^2 B^2$.
\item {\bf{Bremsstrahlung} (Promieniowanie hamowania) }:
kiedy naładowana cząstka jest hamowana w polu elektrycznym emituje fotony. Prawdopodobieństwo oddziaływania $\phi$ zależy od ładunku $z$, masy $m$ i energii $E$ uginanej cząstki oraz liczby atomowej $Z$ ośrodka: $\phi \sim \frac{z^2 Z^2 E}{m^2}$. Z powodu małej masy elektronów, głownie te cząstki biorą udział w procesie Bremsstrahlung. Widmo fotonów produkowanych w procesie jest ciągłe oraz maleje jak $1/E_{\gamma}$ dla wysokich energii.  
\item {\bf{Odwrotny efekt Comptona}}:
relatywistyczny elektron przyspieszany w źródle, oddziałuje z fotonami tła, w wyniku czego przekazuje im część energii.
\item {\bf{Rozpad $\pi^0$}}: proton przyspieszany w źródle może wyprodukować naładowane oraz neutralne piony w procesach oddziaływania proton-proton oraz proton-jądro:
\begin{equation}
p + \mathtt{jadro} \rightarrow p' + \mathtt{jadro'} + \pi^+ + \pi^- + \pi^0
\end{equation} 
Czas życia naładowanych pionów to 26\,ns, po których rozpadają się na miony, natomiast neutralny pion prawie natychmiast rozpada się w dwa kwanty promieniowania gamma:
\begin{equation}
\pi^0 \rightarrow \gamma + \gamma
\end{equation}
Których energia w układzie środka masy wynosi 72.5\,MeV.
\end{itemize}
\subsection{Mechanizmy detekcji promieniowania Gamma}
Procesy fizyce, na których opiera się detekcja fotonów $\gamma$ to \cite{astro_particle}:
\begin{itemize}
\item {\bf{Efekt fotoelektryczny}}, gdy $E_{\gamma} \leq 100$\,keV
\begin{equation}
\gamma + \mathtt{atom} \rightarrow \mathtt{atom^+} + \mathtt{e^-}
\end{equation}
\item {\bf{Efekt Comptona}}, gdy $E_{\gamma} \leq 1$\,MeV
\begin{equation}
\gamma + \mathtt{e^{-}_{w\,spoczynku}} \rightarrow \gamma' + \mathtt{e^{-}_{w\,ruchu}}
\end{equation}
\item {\bf{Produkcja par elektron-pozyton}}, gdy $E_{\gamma} >> 1$\,MeV
\begin{equation}
\gamma + \mathtt{jadro} \rightarrow \mathtt{e^+} + \mathtt{e^-} + \mathtt{jadro'}
\end{equation}
\end{itemize}
Ten ostatni proces jest szczególnie ważny w opisie rozwoju kaskady wywołanej przez wysokoenergetyczny foton $\gamma$. Co zostanie opisane w następny punkcie.
\subsection{Promieniowanie Czerenkowa i wielkie pęki atmosferyczne}
Podstawą działania teleskopów czerenkowskich jest detekcja promieniowanie Czerenkowa powstającego w wielkim pęku atmosferycznym na skutek oddziaływania fotonu gamma z atomami tworzącymi atmosferę ziemską. 
Foton gamma o bardzo wysokiej energii oddziałując z ziemską atmosferą tworzy parę elektron-pozyton. Następnie zachodzące naprzemiennie procesy Bremsstrahlung (promieniowanie hamowania) oraz produkcji par prowadzą do generacji {\bf{kaskady elektromagnetycznej}} w atmosferze. Długość radiacyjna $X_0$ dla wysokoenergetycznych fotonów w procesie Bremmstrahlung wynosi 37,15 $\mathtt{g} \cdot \mathtt{cm^{-2}}$ \cite{IACT}, co odpowiada 7/9 drogi swobodnej dla produkcji par $\mathtt{e^{\pm}}$. Te uproszenia są fundamentem dla łatwej analitycznej formuły opisania pęków atmosferycznych. Całkowita liczba elektronów, pozytonów i fotonów jest podwajana co $\ln(2) X_0$. Pierwotna energia fotonu gamma $E_0$ dzieli się równo między cząstki wtórne. Proces trwa, aż średnia energia elektronów produkowanych w kaskadzie osiągnie energią krytyczną $E_c = 84$\,MeV, a sam pęk na wysokości na jakiej to ma miejsce osiąga swoje maksimum. Dalej   dominują już straty jonizacyjne i kaskada powoli się wygasza \cite{IACT}. Maksymalna liczba cząstek wyprodukowanych w takiej kaskadzie dana jest zależnością $E_0 / E_c$. Schematyczny rozwój kaskady elektromagnetycznej przedstawiam na rysunku~\ref{fig:cas_em}. 
\begin{figure}[H] 
\centering
\includegraphics[scale=0.3]{rysunki/kaskada_em.png}
\caption{Rozwój kaskady elektromagnetycznej.}
\label{fig:cas_em}
\end{figure}
Wielkie pęki są wywoływane także przez promieniowanie kosmiczne (relatywistyczne protony lub jądra atomowe), nazywane {\bf{kaskadą hadronową}}. W tym przypadku rozwój kaskady jest bardziej skomplikowany, procesy hadronowe prowadzą do powstania wtórnych jąder atomowych wraz z neutralnymi i naładowanymi pionami z dużym pędem poprzecznym. Piony są cząstkami o krótkim czasie życia i nie docierają do powierzchni ziemi: neutralny pion rozpada się w dwa fotony $\gamma$, a naładowany pion w mion i neutrino:
\begin{equation*}
\pi^0 \rightarrow \gamma + \gamma 
\end{equation*}
\begin{equation*}
\pi^+ \rightarrow \mu^+ + \nu_{\mu}
\end{equation*}  
\begin{equation*}
\pi^- \rightarrow \mu^- + \bar{\nu}_{\mu}
\end{equation*}
Rożnice w morfologi pęku wraz z rekonstrukcją kierunku przyjścia pierwotnej cząstki, pozwalają na uzyskania wydajnego sposobu na odrzucenie obrazów produkowanych przez hadrony. Nawet dla silnych źródeł promieniowania gamma ilość kaskad hadronowych przewyższa o 3 rzędy wielkości ilość pęków zainicjowanych przez fotony gamma. Symulacje Monte Carlo wielkiego pęku kosmicznego zapoczątkowanego przez foton~\ref{fig:f1} oraz przez proton~\ref{fig:f2} pokazuje na rysunku poniżej.
\begin{figure}[H]
\label{mc_pek}
  \begin{center}
  \subfloat[Foton o energii 1\,TeV.]{\includegraphics[scale=0.15]{obrazy_teoria/foton_1TeV.png}\label{fig:f1}}
  \hfill
  \subfloat[Proton o energii 1\,TeV.]{\includegraphics[scale=0.15]{obrazy_teoria/proton_1TeV.png}\label{fig:f2}}
  \caption{Symulacja Monte Carlo pęku kosmicznego za pomocą programu CORSIKA. Typ cząstki reprezentowany jest przez kolor śladu: czerwony = elektrony, pozytony, fotony gamma; zielony = miony, niebieski = hadrony. Parametry użyte do symulacji to: wysokość pierwszej interakcji = 30\,km, 
zakres na osiach: Z(pionowa): 0 - 30.1\,km, X/Y: $\pm$ 5\,km wokół osi pęku, cięcia energii: 0.1\,MeV ($\mathtt{e^{-+}}$, fotony gamma) oraz 0.1\,GeV (miony i hadrony) \cite{monte_carlo}.}
\end{center}
\end{figure}
Relatywistyczna cząstka poruszająca się w powietrzu szybciej niż prędkość światła w tyn ośrodku ($v > c/n_{air}$, gdzie $n_{air}$ jest współczynnikiem załamania dla powietrza) produkuje {\bf{promieniowanie Czerenkowa}}. Światło Czerenkowa ma swoje maksimum w zakresie UV (ciągnie się aż do zakresu optycznego) oraz jest produkowane przez cały czas rozwoju kaskady, maksimum emisji występuje, kiedy liczba wyprodukowanych cząstek w kaskadzie jest największa, na wysokość $\approx$ 10\,km dla fotonów gamma o pierwotnej energii od 100\,GeV do 1\,TeV \cite{IACT}. Każda cząstka produkuje światło Czerenkowa o stałym kącie $\theta_C$ do kierunku ruchu, co wyrażamy zależnością:
\begin{equation}
\cos (\theta_C) = \frac{c}{v n_{air}}
\end{equation}
Kąt Czerenkowa wynosi $\approx \ang{1.3}$ na poziomie morza. \\
Światła Czerenkowa produkowanego w kaskadzie jest mało (w przybliżeniu ilość światła proporcjonalna jest do energii cząstki pierwotnej), dlatego trzeba używać dużych teleskopów oraz możliwie redukować szum w elektronice systemu odczytu danych do wydajnej jego obserwacji. \\
%do tego momentu Julian sprawdzil tekst 
Cząstki w kaskadzie elektromagnetycznej podlegają także wielokrotnemu rozpraszaniu wywołanego przez oddziaływanie Coulomba, co prowadzi do rozdzielenia ich kierunku propagacji w małym zakresie kątowym generując zasięg boczny pęku. W wyniku powstaje tzw. \textsl{light pool} o promieniu $\approx$ 130\,m oraz gęstości $\approx$ 100 fotonów $\mathtt{m^{-2}}$ dla fotonów gamma o energii 1\,TeV \cite{IACT}, co jest pokazane na rysunku \ref{fig:lightpool}.
\begin{figure}[H] 
\centering
\includegraphics[scale=0.5]{obrazy_teoria/lightpool.png}
\caption{Symulacje Monte Carlo rozkładu promieniowania Czerenkowa na powierzchni ziemi dla kaskady zainicjowanej przez foton gamma. Lewa strona: wykres zależności gęstości promieniowania Czerenkowa w funkcji odległości radialnej od rdzenia pęku dla różnych energii pierwotnych fotonów. Prawa strona: 2-wymiarowy rozkład gęstości fotonów na powierzchni ziemi dla pęku zainicjowanego przez foton o energii 300\,GeV. Wykres dzięki uprzejmości G. Maier \cite{IACT}.}
\label{fig:lightpool}
\end{figure}
Wydajność produkcji światła Czerenkowa jest proporcjonalne do 1/$\lambda^2$ ($\lambda$ --- długość fali), dlatego spektrum powstałego promieniowania jest zdominowane przez emisje w zakresie niebieskim/UV, mające maksimum w okolicy 340\,nm. Krótsze długości fali są absorbowane (głównie przez ozon). Fotony Czerenkowskie z pęku przebywają jako krótkie pulsy światła o długości kilku nanosekund, co wymaga elektroniki która jest w stanie próbkować sygnały z takimi częstotliwościami.
\subsection{Technika IACT (Imaging Atmospheric Cherenkov Telescope)}
Do detekcji fotonów gamma o bardzo wysokich energiach najczęściej używana jest technika IACT, czyli teleskopy Czerenkowskie. Technika ta polega na zbieraniu światło Czerenkowa produkowanego w kaskadzie na dużym lustrze, które pod odbiciu jest rejestrowane przez kamerę. Kamera składa się z wielu fotopowielaczy, które zazwyczaj charakteryzują się sprawnością 30\% \cite{particle_de_angelis}. Średnica kamery to zazwyczaj 1\,m. Sygnał z kamery jest transmitowany w postaci analogowej do systemu wyzwalania (ang. \textsl{trigger systems}). Przypadki, które wyzwolą sygnał są wysyłane do systemu zbierania danych, który zazwyczaj operuje z częstotliwości kilkuset Hz. Typowa rozdzielczość czasu przyjścia impulsu do fotopowielacza jest mniejsza niż 1\,ns. Czas rejestracji kaskady to kilka nanosekund (2-3\,ns).
\begin{figure}[H] 
\centering
\includegraphics[scale=0.1]{obrazy_teoria/iact.pdf}
\caption{Schemat działania teleskopu Czerenkowskiego.}
\label{fig:kamera}
\end{figure}
Jak było już wspomniane wcześniej, tylko około 10 fotonów Czerenkowa na metr kwadratowy dociera do teleskopu z pęku wywołanego przez foton gamma o energii 100\,GeV. Typowa powierzchnia z której zbierane jest światło Czerenkowa to 100\,$\mathtt{m^2}$ na wysokości około 2000\,m.n.p.m \cite{particle_de_angelis}. Z powodu słabego sygnału, obserwacje są prowadzone w bezksiężycowe noce (ewentualnie przy słabym świetle księżyca), bez chmur. Z tego powodu całkowity czas obserwacji to około 1200\,h na rok.

Jak było już wspomniane wcześniej, w zakresie energii GeV-TeV, tło od naładowanych cząstek jest o trzy rzędy wielkości większe niż sygnał od fotonów gamma. Kształt obrazu od kaskady hadronowej jest inny, dzięki czemu możemy odseparować kaskady wywołane przez fotony gamma od tych wywołanych przez hadrony. Obrazy jakie obserwuje w kamerze pokazuje na rysunku \ref{fig:kamera}.
\begin{figure}[H] 
\centering
\includegraphics[scale=0.25]{obrazy_teoria/tlo.pdf}
\caption{Obrazy w kamerze wywołany przez z lewej strony: foton gamma, z prawej strony: hadron.}
\label{fig:kamera}
\end{figure}
Większość obecnej techniki identyfikacji kaskady polega na metodzie opracowanej przez Michaela Hillas w latach 80-tych XX wieku poprzez wyznaczenie tzw. parametrów Hillasa \cite{hilas}.

Struktura czasowa przyjścia impulsów światła Czerenkowa również jest używana w odróżnieniu obrazów wyprodukowanych przez leptony i hadrony. 

Do obserwacje pęków używany jest system kilku teleskopów Czerenkowski, co najmniej dwa teleskopy muszą obserwować pęk.  Zapewnia to lepszą możliwość odrzucanie tła, lepszą rozdzielczość kątową oraz lepszą rozdzielczość energetyczną niż jeden teleskop.

W tym momencie istnieją trzy główne eksperymenty, które używają techniki IACT do detekcji fotonów bardzo wysokich energii, to H.E.S.S, MAGIC i VERITAS, pierwszy znajduje się na półkuli południowej, natomiast dwa ostatnie na półkuli północnej \cite{particle_de_angelis}. 
\begin{itemize}
\item {\bf{H.E.S.S}} znajduje się w Namibii, w skład obserwatorium wchodzą 4 teleskopy o średnicy 12\,m, pracujące od 2003 roku. Piąty teleskop o średnicy około 25\,m jest ulokowany w centrum od 2012 roku.
\item {\bf{MAGIC}} znajduje się na wyspach Kanaryjskich na wyspie La Palma. Jest to system dwóch parabolicznych teleskopów, każdy o średnicy 17\,m oraz o powierzchni 236\,$\mathtt{m^2}$. Pracuje od 2004 roku.
\item {\bf{VERITAS}} znajduje się w Stanach Zjednoczonych w Arizonie, składa się z czterech teleskopów o średnicy 12\,m. Działa od 2007 roku.
\end{itemize}
Typowa czułość teleskopów H.E.S.S MAGIC i VERITAS jest taka, że źródło o  strumieniu światła mniejszym niż 1\% emisji z mgławicy Crab będzie wykryte na poziomie znaczności statystycznej 5$\sigma$ podczas obserwacji trwającej 50 godzin \cite{particle_de_angelis}. 
\subsection{CTA}
Obserwatorium {\bf{Cherenkov Telescope Array}} jest następną generacją teleskopów do obserwacji promieniowania gamma bardzo wysokich energii. Oczekujemy, że czułość CTA będzie o rząd wielkości większa niż czułość obecnych teleskopów. 
Zespół dziesiątek teleskopów będzie obserwował kaskady elektromagnetyczne zainicjowane przez promieniowanie gamma na dużej większej powierzchni niż obecnie, zwiększając efektywność oraz czułość teleskopów przez obserwacje kaskady przez większą ilość teleskopów. Dzięki temu poprawiona będzie rozdzielczość kątowa oraz poprawiona efektywność eliminacji tła od hadronów.

Obserwatorium będzie znajdować się na dwóch półkulach w Hiszpanii i w Chile. W skład obserwatorium będą wchodzić trzy typy teleskopów o różnej średnicy $d$ lustra:
\begin{itemize}
\item {\bf{Mały} ang. \textsl{Small-Sized Telescope (SST)}} ma być ich 70, $d = 4$\,m przeznaczonych do detekcji pęków zapoczątkowanych przez fotony od 5\,TeV do 300\,TeV.
\item {\bf{Średni} ang. \textsl{Medium-Sized Telescope (MST)}} $d = 11.5$\,m. Ma być ich 25 na półkuli południowej i 15 na półkuli północnej. Przewidywany zakres energii to 150\,GeV -- 5\,TeV.
\item {\bf{Duży} ang. \textsl{Large-Sized Telescope (LST)}} mają być 4 zarówno na półkuli południowej jak i północnej, $d = 23\,m$. Inauguracja pierwszego teleskopu LST miała miejsce w październiku 2018 roku. 
\end{itemize}
Na półkuli południowej w Chile będą wszystkie 3 typy teleskopów,
 aby pokryć cały zakres energii jaki będzie celem CTA, czyli od 20\,GeV do 300\,TeV \cite{cta_web}. Natomiast na półkuli północnej w Hiszpanii będą znajdować się twa typy teleskopów: duży i średni  

\subsection{Analiza danych IACT}
Celem analizy danych jest identyfikacja rodzaju cząstki wytwarzającej  kaskadę, której obraz rejestrujemy, wyznaczenie jej kierunku przyjścia oraz energii. Następnie ta informacja jest używana do obliczenie znaczności statystycznej badanego obszaru nieba (czy znajduje się tam źródło fotonów gamma), wyznaczenia :rozkładu sygnału na badanym obszarze, strumienia fotonów gamma oraz widma energetycznego. 

Surowy sygnał z teleskopów czerenkowskich pochodzący z systemu akwizycji danych zawiera cyfrowo próbkowany ślad sygnału dla każdego fotopowielacza wchodzącego w skład kamery. Wzrost sygnału pochodzi od fotonów czerenkowa, co przybliżeniu jest czasem przyjścia impulsu jak pokazane jest na rysunku [RYSUNEK].

\begin{figure}[H] 
\centering
\includegraphics[scale=0.45]{obrazy_teoria/wykresy/sygnal_teoria.pdf}
\caption{Schemat działania teleskopu Czerenkowskiego.}
\label{fig:kamera}
\end{figure}

Pierwszym krokiem w analizie sygnału jest zmierzenie i odjęcie linii bazowej, czyli wartości sygnału podczas nieobecności światła Czerenkowa. W celu uzyskania sygnału od kaskady całkuje się obszar pod krzywa używając algorytmu "okno przesuwne" (ang. \textsl{sliding window}) poprzez poszukiwanie maksymalnej wartości sumy ustalonej liczby próbek sygnału.

Praca ta koncentruje się głównie na tym pierwszym etapie analizy danych, których celem jest redukcja szumów w elektronice systemu odczytu danych teleskopu LST. 

\newpage
\section{System odczytu sygnału teleskopu LST}
Teleskop LST  ma lustro o średnicy 23 metrów, a kamera LST składa się z 1855 fotopowielaczy (PMT). Ze względu na dużą powierzchnie lustra, fotony tła nocnego nieba (ang. \textsl{night sky background}, NSB) również są rejestrowane z częstotliwością kilkuset MHz na piksel. W celu redukcji zanieczyszczenia sygnału przez NSB oraz poprawy stosunku sygnału do szumu, konieczne jest próbkowanie  sygnału z szybkością $\sim$ 1 GHz i zmniejszenie czasu całkowania sygnału do kilku ns \cite{dragon_lst}.
\subsection{Wprowadzenie}
Kamera LST składa się z 265 modułów, a każdy moduł składa się z 7 pikseli, co daje nam w sumie 1855 pikseli (fotopowielaczy), które tworzą kamerę LST. Jeden moduł (pokazany na rysunku \ref{fig:drs4}) składa się m.in. z chipu DRS4 będącym sercem układu odczytu sygnału, 7 fotopowielaczy(PMT), przedwzmacniacza.
Z każdego piksela danego są odczytywane za pomocą jednego kanału chipu DRS4. W każdym module, sygnał z przedwzmacniacza jest dzielony na trzy linie: linię wysokiego wzmocnienia, linię niskiego wzmocnienia i linię wyzwalania (ang. \textsl{trigger}). Sygnały o wysokim wzmocnieniu (ang. \textsl{high gain}) i niskim wzmocnieniu (ang. \textsl{low gain}) są próbkowane przez układ DRS4 z częstotliwością 1\,GHz. Gdy następuje wyzwolenie sygnału, przebieg sygnału (ang. \textsl{waveform}) jest digitalizowany przez zewnętrzy przetwornik analogowo-cyfrowy (ADC) z częstotliwością $\sim$ 33\,MHz. Sygnał w postaci cyfrowej jest przesyłany do bramek programowalnych (FPGA), następnie do serwera, gdzie dane są przechowywane \cite{dragon_lst}. 
\begin{figure}[H] 
\centering
\includegraphics[scale=0.35]{rysunki/modul.png}
\caption{Jeden moduł systemu odczytu danych teleskopu LST.}
\label{fig:drs4}
\end{figure}
\subsection{Chip DRS4}
%System odczytu danych jest częścią łańcucha przetwarzania danych, który digitalizuje i tymczasowo przechowuje sygnał analogowy z teleskopu (kamery). 
Chip DRS4 (Domino Ring Sampler w wersji 4) został stworzony w Paul Scherrer Institute (PSI) w Szwajcarii do eksperymentu MEG \cite{meg_experiment}.
Znajduje on także zastosowanie w eksperymentach w astronomii gamma, używają go eksperymenty MAGIC i FACT \cite{drs4_magic}.

Układ DRS4 ma dziewięć kanałów wejściowych o szerokości pasma 950\,MHz, a każdy kanał ma 1024 kondensatorów pamięci, których przełączniki zapisu są obsługiwane przez pierścień zwany \textsl{domino wave circuit}. Prędkość próbkowania można zmieniać w zakresie 700\,MHz do 5\,GHz za pomocą zegara referencyjnego z FPGA. Osiem chipów DRS4 wchodzi w skład układ odczytu danych, a każdy chip jest podłączony do dwóch kanałów fotopowielacza (PMT). Możliwe jest kaskadowe połączenie do ośmiu kanałów DRS4 w celu uzyskania głębszego próbkowania sygnału. W systemie odczytu, używanym przez teleskop LST cztery kanały DRS4 są kaskadowane, co oznacza, że sygnał każdego wejścia jest próbkowany przez 4096 kondensatorów, co daje głębokość próbkowania $\sim 4 \mathtt{\mu}$s w przypadku próbkowania z częstotliwością 1\,GHz \cite{dragon_lst}.

Zasada działania chipu DRS4 jest następująco: sygnał wychodzący z systemu odczytu jest sekwencyjne połączony z pierścieniem (układem) 1024 kondensatorów poprzez szybkie przełączniki zsynchronizowane z zegarem zewnętrznym. Na każdym z kondensatorów jest napięcie wytworzone przez sygnał analogowy z danego piksela w funkcji czasu, który jest proporcjonalny do okresu zegara sterującego przełącznikiem (tzw. fala Domino). Kondensatory są nadpisywane po 1024 cyklach zegara. Po wyzwoleniu sygnału fala Domino zatrzymuje się i ładunek zostaje zapisany na 1024 kondensatorach, następnie jest digitalizowany przez przetwornik analogowo-cyfrowy (ADC). Później tylko 40 
kondensatorów(próbek) jest przechowywane przez system akwizycji danych (DAQ) \cite{drs4_psi}. 

Chip DRS4 wymaga szeregu korekcji programowalnych w celu optymalnej pracy.
\subsection{Procedury korekcji w DRS4}
\subsubsection{Korekcja linii bazowej dla przypadków ze stałym czasem przyjścia}
Każdy kondensator w chipie DRS4 ma swoją własną wartość szumów. Na rysunku pokazana jest średnią wartość bazową (szumów) oraz ich RMS w funkcji absolutnej pozycji w pierścieniu domino \ref{fig:baseline_cap}. Rożna wartość bazowa danego kondensatora spowodowana jest fizyczną różnicą magazynowania ładunku w każdym kondensatorze, która jest większa niż jego RMS. Wyraźny skok na pozycji 512 spowodowany jest wewnętrzną konstrukcją chipu DRS4. Wartość bazowa każdego kondensatora musi być skalibrowana w celu osiągnięcia niskich wartości szumów elektroniki. W tym celu stosuje się prostą kalibracje polegającą na obliczeniu średniej wartości bazowej (szumu) w funkcji numeru kondensatora z danych pochodzących z pomiaru kalibracyjnego linii bazowej (tzw. \textsl{pedestal run}). Ta obliczona średnia wartość bazowa może być następnie odjęta z sygnału dla danego kondensatora \cite{drs4_magic}. 
\begin{figure}[H] 
\centering
\includegraphics[scale=0.25]{wykresy/drs4/pedestal_teoria.pdf}
\caption{Wartość bazowa 1024 kondensatorów jednego kanału chipu DRS4 w funkcji numeru kondensatora.}
\label{fig:baseline_cap}
\end{figure}
\subsubsection{Korekcja czasowa linii bazowej dla przypadków z losowym czasem przyjścia}
Nawet po zastosowaniu kalibracji opisanych w poprzedniej sekcji, linia bazowa nie jest w pełni stabilna dla wyzwalacza z losowym czasem przyjścia. Jest to spowodowane zależność linii bazowej od czasu odczytu danego kondensatora. Może się zdarzyć, że dany kondensator jest odczytywany w krótkim odstępie czasu, wtedy dochodzi do skoku sygnału na tym kondensatorze \cite{drs4_magic}. Przykładowe przebiegi sygnału kiedy następuje skok sygnału przedstawiony jest na rysunku \ref{fig:dt_corr}. 
\begin{figure}[H] 
\centering
\includegraphics[scale=0.55]{wykresy/drs4/dt_corr_waveform.pdf}
\caption{Przykładowe przebiegi sygnału w których wystąpił skok sygnału spowodowany krótkim czasem odczytu danego kondensatora.}
\label{fig:dt_corr}
\end{figure}
W celu korekcji sygnału używa się funkcji potęgowej w postaci:
\begin{equation}
\label{eqn:power_law}
y = A \cdot dt^B + C
\end{equation}
gdzie: $dt$ --- czas pomiędzy odczytem danego kondensatora; $A, B, C$ --- współczynniki wyznaczone eksperymentalnie. 
Współczynniki te zależą głównie od temperatury w jakiej pracuje elektronika. Na rysunku \ref{fig:dt_corr_fit} pokazana jest funkcja potęgowa dopasowana do danych w różnych temperaturach, dane pochodzą z testów Japonii. Jednym z moich zadań w tej pracy jest sprawdzenie jakie są współczynniki w układzie działającym w LST, oraz czy są one różne dla różnych pikseli. 
\begin{figure}[H] 
\centering
\includegraphics[scale=0.45]{wykresy/drs4/dtcurve.pdf}
\caption{Funkcja potęgowa używana do korekcji skoku w linii bazowej, dopasowana do danych w różnych temperaturach. Wykres od Seiya Nozaki.}
\label{fig:dt_corr_fit}
\end{figure}
\newpage
\subsubsection{Interpolacja szumów niegaussowskich}
Szumy niegaussowskie spowodowane  są skokiem napięcia podczas odczytu kondensatorów, występują one kiedy pierścień domino, z którego odczytywane są dane, zatrzymana się w wybranych miejscach w następujących po sobie zdarzeniach (ang. \textsl{event}). Występują dwa typy szumów niegaussowskich:
\begin{itemize}
\item {\bf{typu A}} jest to większy skok sygnału $\sim 50 - 80$ jednostek ADC, o szerokości dwóch próbek.
\item {\bf{typu B}} jest to mniejszy skok sygnału $\sim 20$ jednostek ADC, o szerokości jednej próbki.
\end{itemize}
Na rysunku \ref{fig:spike_corr} pokazany jest przebieg sygnału, gdzie występuje szum niegaussowski typu A i B. 
\begin{figure}[H] 
\centering
\includegraphics[scale=0.55]{wykresy/drs4/spikes_teoria.pdf}
\caption{Przebieg sygnału w którym wystąpił szum niegausowski typu A (lewy) i B(prawy).}
\label{fig:spike_corr}
\end{figure}
\subsubsection{Korekcja czasowa}
Innym, ważnym rodzajem korekcji, które należy zastosować w DRS4 jest korekcja czasu przyjścia impulsu. W chipie DRS4 występuje opóźnienie typowe to 1\,ns (do 4\,ns), zależne od absolutnej pozycji impulsu w pierścieniu domino. Rożnicę w czasie przyjścia mogą być skalibrowane za pomocą impulsów kalibracyjnych. W tym celu oblicza się średni czas przyjścia impulsu w funkcji pozycji impulsu w pierścieniu domino. Zależność ta jest różna dla różnych kanałów w chipie DRS4. Dla każdego kanału wykorzystywane jest rozwinięcie w szereg fouriera, aby uzyskać funkcje (współczynniki) korekcji \cite{drs4_magic}. 
\begin{figure}[H] 
\centering
\includegraphics[scale=0.45]{wykresy/drs4/time_corr.pdf}
\caption{Kalibracja czasu przyjścia impulsów. Po lewej stronie : średni czas przyjścia w funkcji pozycji w pierścieniu DRS4 (punkty) wraz z rozwinięciem w szereg fouriera (czerwona krzywa). Po prawej stronie: rozkład czasów przyjścia przed (przerywana linia) i po (ciągła linia) kalibracji. Źródło: \cite{drs4_magic}. }
\label{fig:spike_corr}
\end{figure}
Korekcja ta jest ważna do wyznaczenie ewolucji czasowej rozwoju pęku, co może być następnie użyte m.in. do separacji obrazów wytworzonych przez fotony gamma/leptony a hadrony. 
\subsubsection{Implementacja korekcji}
W ramach pracy zaimplementowane wyżej opisane korekcje w bibliotece cta-lstchain. Bibliotek cta-lstchain \cite{lstchain_url} jest biblioteką napisaną w języku Python do analizy danych z teleskopu LST. Biblioteka ta jest oparta o bibliotekę ctapipe \cite{ctapipe_url}, będącą podstawowym narzędziem do niskopoziomowej analizy danych z teleskopów CTA.


\newpage
\section{Zastosowanie korekcji}
W tym rozdziale pokaże zastosowanie korekcji w praktyce do danych testowych zbieranych w obserwatorium Roque de los Muchachos w Hiszpanii, gdzie znajduje się teleskop LST. Dane były zbierane w okresie listopad 2018 -- marzec 2019. Ich celem było przetestowanie jak działa elektronika, system akwizycji danych, jak działają korekcje. Pokaże rozkłady sygnału przed i po dokonaniu korekcji, pokaże jak poszczególne korekcji zmniejszają szum sygnału. Korekcje są niezbędne w celu wydobycia użytecznego sygnału z danych, co będzie pokazane w dalszych rozdziałach. W tym rozdziale pokazuje zastosowanie korekcji do danych testowych: \textsl{Run00097}, których celem był pomiar szumów elektroniki. Na rysunku \ref{fig:data_hist} pokazuje histogram czasu między zdarzeniami (ang. \textsl{events}) oraz pokrycie pierwszego kondensatora. Z przedstawionych histogramów wynika, że wszystkie kondensatory były pokryte w tym pomiarze oraz że zdarzenia przychodziły z losowym czasem, czyli wyzwalacz (ang. \textsl{trigger}) był losowy. 
\begin{figure}[H] 
\centering
\includegraphics[scale=0.36]{wykresy/drs4/hist_fc_arrival_time.pdf}
\caption{Histogram pokrycia kondensatorów oraz czasu przyjścia zdarzeń z dopasowaną funkcją eksponencjalną. Run 00097, wszystkie piksele.}
\label{fig:data_hist}
\end{figure}
\subsection{Korekcja linii bazowej}
Aby zastosować korekcje linii bazowej, potrzebujemy znać średnią wartość bazową każdego kondensatora. W ramach pracy napisałem skrypt do tworzenia pliku, który zawiera te wartości dla każdego kondensatora wchodzącego w układ systemu odczytu sygnału. Wielkość tego pliku to $\sim 30$\,MB, zawarte w nim są wartość dla 1855 pikseli, każdy piksel ma dwa wzmocnień i składa się z 4096 kondensatorów. Plik jest tworzony na podstawie specjalnego pomiaru tzw.\textsl{pedestal run}. Na rysunku \ref{fig:ped_4096} pokazuje średnią wartość bazową w funkcji absolutnej pozycji w pierścieniu domino dla jednego piksela. Skok co 512 kondensatorów wynika z wewnętrznej konstrukcji chipu DRS4, co było wspomniane już w rozdziale 2.
\begin{figure}[H] 
\centering
\includegraphics[scale=0.3]{wykresy/drs4/ped_hg.pdf}
\caption{Średnia wartość bazowa w funkcji absolutnej pozycji dla jednego piksela wraz z obliczoną wartość RMS.}
\label{fig:ped_4096}
\end{figure}
Wartość błędu wartość bazowej każdego kondensatora możemy oszacować za pomocą policzenia RMS (ang. \textsl{Root Mean Square}). RMS rozumiem jako:
\begin{equation}
RMS = \sqrt{ \frac{1}{N} \sum_{i=1}^{N} x_{i}^2 - \mu^2}
\end{equation}
,gdzie $x_{i}$ --- dana wartość, $\mu$ --- średnia.
Wartość RMS zależy od wzmocnienia kanału, generalnie dla niskiego wzmocnienia sygnału (ang. \textsl{low gain}) będzie ono mniejsze niż dla wysokiego wzmocnienia kanału (ang. \textsl{high gain}). Histogram wartości RMS dla low i high gain dla jednego piksela pokazuje na rysunku \ref{fig:hist_rms}.
\begin{figure}[H] 
\centering
\includegraphics[scale=0.2]{wykresy/drs4/hist_rms.pdf}
\caption{Histogram wartość błędu wartości szumu liczonego za pomocą RMS dla dwóch wzmocnień.}
\label{fig:hist_rms}
\end{figure}
%W celu uzyskanie wartości szumów dla każdego kondensatora wchodzącego w skład piksela, bierze się tzw. pedestal run, w którym zbiera się wartości sygnału/szumów na danym kondensatorze. 
%W ramach pracy napisałem program do tworzenia pliku z wartościami szumu, dostępny w bibliotece cta-lstchain, opartej na bibliotece ctapipe, będącej biblioteką do niskopoziomowych kalibracji teleskopów CTA. 
\newpage
Kiedy znamy średnią wartość bazową każdego kondensatora, stosujemy pierwszą kalibracje polegającą na odjęciu tej wartość dla danego kondensatora w odczytanym sygnale. Przebieg sygnału po zastosowaniu tej korekcji pokazuje na rysunku \ref{fig:baseline_corr}. W celu unikania negatywnych wartości dodaje do sygnału wartość 300. Na rysunku \ref{fig:baseline_corr_hist} pokazuje rozkład sygnału dla wszystkich pikseli po odjęciu średniej wartości bazowej. Jako miarę sygnału używam odchylenie standardowe $\sigma$ sygnału liczone za pomocą funkcji $\mathtt{numpy.sqrt}$.
\begin{figure}[H] 
\centering
\includegraphics[scale=0.3]{wykresy/drs4/baseline_corr.pdf}
\caption{Korekcja linii bazowej. Przebieg sygnału dla dwóch wzmocnień.}
\label{fig:baseline_corr}
\end{figure}

\begin{figure}[H] 
\centering
\includegraphics[scale=0.45]{wykresy/drs4/baseline_corr_hist2.pdf}
\caption{Korekcja linii bazowej histogram sygnału dla dwóch wzmocnień, wszystkie piksele. Jako miarę sygnału używam odchylenie standardowe sygnału $\sigma$.}
\label{fig:baseline_corr_hist}
\end{figure}

\newpage
\subsection{Korekcja czasowa linii bazowej}
Na rysunku \ref{fig:dt_corr_Waveform} pokazuje przebieg sygnału, gdzie występuje skok sygnału spowodowany małym odstępem czasu odczytu danego kondensatora. W celu korekcji sygnału używał funkcji potęgowej, której argumentem jest czas, który upłynął od ostatniego odczytu tego samego kondensatora.  
\begin{figure}[H] 
\centering
\includegraphics[scale=0.35]{wykresy/drs4/dt_corr_teoria.pdf}
\caption{Przykładowe przebiegi sygnału w których wystąpił skok sygnału spowodowany krótkim czasem odczytu tego kondensatora.}
\label{fig:dt_corr_Waveform}
\end{figure}
\subsubsection{Funkcja potęgowa do kalibracji}
Jak było napisane w poprzednim rozdziale do kalibracji wykorzystuje się funkcje potęgową \ref{eqn:power_law}, której współczynniki zależą od temperatury. Z testów w Japonii zdane są współczynniki przy temperaturach: 10, 20, 30, 40 i 50 $^\circ$C. W ramach swojej pracy wyznaczyłem  współczynniki dla wszystkich pikseli wchodzących w skład systemu odczytu sygnału działającego w teleskopie LST. Na wykresie \ref{fig:dt_curve_fit_all_temp} pokazuje krzywą korekcji czasowej dopasowaną do danych testowych dla jednego piksela, wykreśliłem także funkcje potęgową używaną do korekcji dla różnych temperatur. Jak widzimy na rysunku, najlepiej pasuje ta ze współczynnikami dla temperatury 30$^{\circ}$. 
\begin{figure}[H] 
\centering
\includegraphics[scale=0.45]{wykresy/drs4/dt_curve_fit.pdf}
\caption{Funkcja potęgowa do korekcji czasowej: dopasowana do histogram 2-d oraz wyznaczona eksperymentalnie dla różnych temperatur. Wykres dla  pojedynczego piksela.}
\label{fig:dt_curve_fit_all_temp}
\end{figure}
Na wykresie na rysunku \ref{fig:dt_curve_fit_few_pixels} pokazuje dopasowaną funkcje potęgową dla kilku pikseli, wraz z wykreśloną funkcją potęgową dla temperatur: 10, 20, 30 $^{\circ}$, także widzimy, że funkcje dla temperatury 30$^{\circ}$ pasuje najlepiej.
\begin{figure}[H] 
\centering
\includegraphics[scale=0.35]{wykresy/drs4/dt_curve_fit_diff.pdf}
\caption{Funkcja potęgowa dopasowana dla kilku pikseli wraz z funkcjami potęgowymi dla temperatur 20, 30, 40$^{\circ}$.}
\label{fig:dt_curve_fit_few_pixels}
\end{figure}
Aby sprawdzić dla wszystkich pikseli jak dopasowane funkcje mają się do funkcji potęgowej dla różnych temperatury, policzyłem różnice w dwóch punktach: 0.05\,ms i 0.45\,ms i sporządziłem histogramy pokazane na rysunku \ref{fig:dt_curve_hist}. Z histogramów jasno wynika, że najlepiej pasuje funkcja potęgowa dla temperatury 30$^{\circ}$.
\begin{figure}[H] 
\centering
\includegraphics[scale=0.25]{wykresy/drs4/dt_curve_fit_diff_hist.pdf}
\caption{Histogram różnicy wartości funkcji potęgowej dopasowanej dla danego piksela, a funkcją potęgowa dla danej temperatury policzona w dwóch punktach.}
\label{fig:dt_curve_hist}
\end{figure}
\subsubsection{Korekcja}
Na rysunku \ref{fig:dt_corr_hist2d} pokazuje histogram 2-d dla jednego piksela, gdzie przedstawiony jest sygnał w funkcji różnicy czasu ostatniego odczytu tego samego kondensatora z używaną funkcją potęgową dla temperatury: 30 $^{\circ}$C. 
\begin{figure}[H] 
\centering
\includegraphics[scale=0.25]{wykresy/drs4/dt_hist2d.pdf}
\caption{Korekcja czasowa linii bazowej. Histogram 2-d dla jednego piksela (15 000 zdarzeń) z używaną funkcją potęgową do korekcji. Na górze przed korekcją, na dole po dokonaniu korekcji.}
\label{fig:dt_corr_hist2d}
\end{figure}

\begin{figure}[H] 
\centering
\includegraphics[scale=0.38]{wykresy/drs4/dt_corr_hist2d.pdf}
\caption{Korekcja czasowa linii bazowej. Histogram 2-d dla wszystkich pikseli dla High gain (15000 zdarzeń) z używaną funkcją potęgową do korekcji. Na górze przed korekcją, na dole po dokonaniu korekcji.}
\label{fig:dt_corr_hist2d_all}
\end{figure}

\begin{figure}[H] 
\centering
\includegraphics[scale=0.38]{wykresy/drs4/dt_corr_lg_hist2d.pdf}
\caption{Korekcja czasowa linii bazowej. Histogram 2-d dla wszystkich pikseli dla Low gain (15000 zdarzeń) z używaną funkcją potęgową do korekcji. Na górze przed korekcją, na dole po dokonaniu korekcji.}
\label{fig:dt_corr_hist2d_all}
\end{figure}

Na rysunku \ref{fig:dt_corr_hist} pokazuje histogram rozkładu sygnału po dokonaniu korekcji czasowej linii bazowej w porównaniu do rozkładu sygnału po pierwszej korekcji linii bazowej. Histogramy zrobiłem dla wszystkich pikseli oraz dla jednego piksela dla wysokiego wzmocnienia (dla niskiego wzmocnienia histogram by wyglądał podobnie). Jak widzimy na histogramach, po korekcji czasowej rozkład staje się bardziej wąski, ale wciąż zostają ogony.  
\begin{figure}[H] 
\centering
\includegraphics[scale=0.45]{wykresy/drs4/dt_corr_hist2.pdf}
\caption{Korekcja linii bazowej histogram sygnału po korekcji czasowej linii bazowej (zielona ciągła linia), po zwykłej korekcji linii bazowej (przerywana pomarańczowa). Po lewej stronie: histogram dla wszystkich pikseli, po prawej stronie: histogram dla jednego piksela. Jako miarę sygnału liczę odchylenie standardowe sygnału $\sigma$: dla wszystkich pikseli: przed korekcją czasową $\sigma = 8.44$, po korekcji czasowej $\sigma = 6.41$. Dla jednego piksela: przed korekcją czasową $\sigma = 8.90$, po korekcji czasowej $\sigma = 6.88$. }
\label{fig:dt_corr_hist}
\end{figure}
\newpage
\subsection{Interpolacja szumów niegaussowskich}
Na rysunku \ref{fig:spike_corr} pokazany jest przebieg sygnału, gdzie występuje szum niegaussowski typu A i B. W celu korekcji sygnału, używa się interpolacji sygnału w miejscu gdzie skok powstał.
\begin{figure}[H] 
\centering
\includegraphics[scale=0.4]{wykresy/drs4/spike_inter.pdf}
\caption{Przebieg sygnału w którym wystąpił szum niegausowski typu A (lewy) i B(prawy). Czerwona przerywana linia przed interpolacją, zielona ciągła linia po.}
\label{fig:spike_corr}
\end{figure}
Na rysunku \ref{fig:spike_hist} przedstawiam histogram po interpolacji szumów niegaussowskich. Jak widzimy na histogramie, ogony po tej korekcji znikają i rozkład przypomina rozkład Gaussa, co jest oczekiwane.
\begin{figure}[H] 
\centering
\includegraphics[scale=0.45]{wykresy/drs4/spike_corr_hist2.pdf}
\caption{Histogram sygnału: po interpolacji szumów niegaussowskich(NG) (czerwona ciągła linia), po zwykłej i czasowej korekcji linii bazowej (zielona przerywana). Po lewej stronie: histogram dla wszystkich pikseli, po prawej stronie: histogram dla jednego piksela.
Jako miarę sygnału liczę odchylenie standardowe sygnału $\sigma$: dla wszystkich pikseli: po korekcji zwykłej i czasowej linii bazowej $\sigma = 6.41$, po interpolacji szumów (NG) $\sigma = 6.21$. Dla jednego piksela: po korekcji zwykłej i czasowej linii bazowej $\sigma = 6.88$, po interpolacji szumów NG $\sigma = 6.72$.}
\label{fig:spike_hist}
\end{figure}
\subsection{Mapa rozkładu sygnału na całej kamerze}
Po opisaniu poszczególnych korekcji, teraz pokaże jaki korekcje wpływają na sygnał na całej kamerze. W tym celu obliczyłem średnie odchylenie sygnału $\sigma$ przed korekcją i po korekcji używając 350 zdarzeń, ilustrując to na całej kamerze, dla  wysokiego wzmocnienia (High Gain), co przedstawiam na rysunku \ref{fig:map_cam_corr}. Na histogramie widzimy, że przed korekcją $\sigma$ wynosi około $\sim$ 30, natomiast po zastosowaniu korekcji $\sigma$ wynosi $\sim$ 6 i ta wartość jest dużo bardziej jednolita na całej kamerze.
\begin{figure}[H] 
\centering
\includegraphics[scale=0.35]{wykresy/drs4/map_corr2.pdf}
\caption{Odchylenie standardowe sygnału na całej kamerze przed i po zastosowaniu korekcji.}
\label{fig:map_cam_corr}
\end{figure}
\subsection{Korekcje w cta-lstchain}
W bibliotece {\bf{cta-lstchain}} w module {\bf{calib/camera/r0.py}} znajduje się klasa $\mathtt{LSTR0Corrections}$ napisana w ramach pracy. Klasa ta ma trzy metody do korekcji:
\begin{itemize}
\item linii bazowej $\mathtt{subtract\_pedestal(event)}$
\item czasowej linii bazowej $\mathtt{time\_lapse\_corr(event)}$
\item interpolacji szumów niegaussowskich $\mathtt{interpolate\_spikes(event)}$
\end{itemize}
oraz skrypt $\mathtt{create\_pedestal\_file.py}$ do tworzenia pliku z wartościami bazowymi każdego kondensatora (wielkość pliku $\sim$30\,MB).
\newpage
\section{Korekcja czasowa}
W tym rozdziale opisuje korekcje czasu przyjścia sygnału. 
W celu dokonania kalibracji czasowej zbierane są dedykowane pomiary kalibracyjne polegające na wstrzykiwaniu impulsów z lasera. Na rysunku \ref{fig:calib_waveform_hg} przedstawiam przebieg sygnału z impulsem kalibracyjnym dla wysokiego wzmocnienia wraz z wyznaczonym czasem przyjścia impulsu. Wyniki prezentowane w tym rozdziale są tylko dla sygnału z wysokim wzmocnieniem, spowodowane to jest, faktem, że impulsy dla niskiego wzmocnienia były słabe i nie było możliwe wykorzystanie tego sygnału do korekcji czasowej.  Na podstawie dostępnych danych obliczam współczynniki kalibracyjny będące współczynnikami w rozwinięciu Fouriera krzywej używanej do kalibracji. W ramach pracy napisałem kod: do wyznaczanie czasu przyjścia impulsu, obliczania współczynników oraz ich zapisywania do pliku (używam formatu pliku {\bf{h5py}}), kalibracji czasu przyjścia impulsu. Wyniki, które tutaj przestawiam oparte są na pomiarach za pomocą tzw. \textsl{calibration box}, które były zbierane 12 marca 2019 roku są to \textsl{run 250 i 252}. Na podstawie danych z \textsl{run 250} (11 500 zdarzeń) obliczyłem współczynniki kalibracyjne, a na danych z \textsl{run 252} (1500 zdarzeń) zastosowałem te korekcje. 
\begin{figure}[H] 
\centering
\includegraphics[scale=0.35]{wykresy/drs4/calib_pulse_waveform_hg.pdf}
\caption{Przebieg sygnału z impulsem kalibracyjnym (czerwona ciągła linia) dla dwóch pikseli dla wysokiego wzmocnienia wraz z wyznaczonym czasem przyjścia (zielona kreskowana).}
\label{fig:calib_waveform_hg}
\end{figure}
Na rysunku \ref{fig:signal_map} pokazuje średni sygnał (który jest sumą 7 próbek czasu w okolicy punktu wyznaczonego jako czas przyjścia impulsu za pomocą {\bf{extractora}} z biblioteki {\bf{ctapipe}}) na całej kamerze dla danych z pomiarów \textsl{run 250 i 252}. Widoczny wzór na kamerze, gdzie sygnał jest bardzo mały spowodowany jest cieniem systemu do kalibracji. W danych z \textsl{run 252} widzimy także, że na dwóch modułach był brak sygnału.
\begin{figure}[H] 
\centering
\includegraphics[scale=0.25]{wykresy/drs4/mapka_sygnal_time_corr.pdf}
\caption{Korekcja linii bazowej histogram, wszystkie piksele.}
\label{fig:signal_map}
\end{figure}
\subsection{Krzywa kalibracyjna}
Na rysunku \ref{fig:fourier_fit_4096} przedstawiam zależność czasu przyjścia impulsu od pozycji pierwszego odczytanego kondensatora w pierścieniu DRS4 dla 4096 kondensatorów wraz z dopasowaną krzywą korzystając z rozwinięcia w szereg Fouriera przy wykorzystaniu 32 harmonicznych.
\begin{figure}[H] 
\centering
\includegraphics[scale=0.35]{wykresy/drs4/fourier_fit.pdf}
\caption{Zależność czasu przyjścia impulsu od pozycji w pierścieniu DRS4 dla 4096 kondensatorów z funkcją dopasowaną przez rozwinięcie w szereg Fouriera przy wykorzystaniu 32 harmonicznych.}
\label{fig:fourier_fit_4096}
\end{figure}
Ze względu, że w systemie odczytu sygnału używany przez LST jeden piksel jest obsługiwany przez 4 kanały tego samego chipu DRS4, a wszystkie kanały mają taką samą krzywą opóźnienia, to wystarczył to odczytaną pozycje pierwszego kondensatora $FC$ w pierścieniu DRS4 liczyć jako modulo 1024 oraz w celu zwiększenia statystyki 8 kondensatorów binuje jako jeden punkt. 
\newpage
Otrzymaną zależność rozwijam w szereg Fouriera (niezupełnie) wtedy
średnią wartość przyjścia impulsu od pozycji w pierścieniu DRS4 zapisujemy jako sumę sinusów i cosinusów
\begin{equation}
y(FC) = \sum_{i=1}^{N} A_n \sin \left( \frac{FC}{1024} \right) + B_n \cos \left( \frac{FC}{1024} \right)
\end{equation}
gdzie $A_n$ i $B_n$ są współczynnikami rozwinięcia funkcji w szereg Fouriera; $N$ --- ilość harmonicznych.
%Na rysunku \ref{fig:fourier_fit_4096} przedstawiam zależność czasu przyjścia impulsu od pozycji w pierścieniu DRS4 dla 4096 kondensatorów. 
%Widzimy, że kształt krzywej zależności się powtarza co 1024 kondensatory. 
%Więc wykorzystując ten fakt podczas kalibracji wartość kondensatora biorę jako modulo 1024 i tak samo krzywe opóźnienie wykreślam dla 1024 kondensatorów. 

Na rysunku \ref{fig:fourier_fit_harm} przedstawiam zależność czasu przyjścia impulsu od położenie pierwszego kondensatora w pierścieniu DRS4 dla dwóch pikseli, z dopasowanymi krzywymi używając 8 i 16 harmonicznych. Jak widzimy na wykresach jedna i druga krzywa dość dobrze pasuje, lecz lepiej pasuje ta tworzona za pomocą 16 harmonicznych (przy czym czas obliczania tych współczynników jest dość krótki) i z tego powodu w dalszej części pracy będę używał 16 harmonicznych.
\begin{figure}[H] 
\centering
\includegraphics[scale=0.45]{wykresy/drs4/fourier_fit_harm.pdf}
\caption{Zależność czasu przyjścia impulsu od pozycji w pierścieniu DRS4 dla 1024 kondensatorów z funkcją dopasowaną przez rozwinięcie w szereg Fouriera przy wykorzystaniu 8 i 16 harmonicznych.}
\label{fig:fourier_fit_harm}
\end{figure}
\newpage
Na rysunku \ref{fig:fourier_fit_3} przedstawiam krzywe opóźnienia dla trzech pikseli, jak widać na rysunkach krzywe są różne dla różnych pikseli. 
\begin{figure}[H] 
\centering
\includegraphics[scale=0.24]{wykresy/drs4/fourier_fit_3pixels.pdf}
\caption{Krzywe opóźnienia dla 3 piksel, dopasowana krzywa (ciągła czerwona linia) powstała przez rozwinięcie w szereg Fouriera przy użyciu 16 harmonicznych.}
\label{fig:fourier_fit_3}
\end{figure}
\subsection{Korekcja czasu przyjścia impulsów}
Mając wyznaczone krzywe opóźnienia stosuje korekcje czasową w celu poprawienia rozdzielczości czasowej czasu przyjścia. Rozróżniamy dwie korekcje czasowe:
\begin{itemize}
\item {\bf{względną}} polegającą na odjęciu czasu z danej krzywej opóźnienia dla danego piksela od wyznaczonego czasu przyjścia impulsu.
\item {\bf{bezwzględną}} polegającą na odjęciu czasu z wyznaczonej krzywej opóźnienia od wyznaczonego czasu przyjścia impulsu z uwzględnieniem średniego czasu przyjścia impulsu we wszystkich pikselach, który sygnał jest powyżej progu.
\end{itemize} 
Na rysunku \ref{fig:time_corr_hist3} pokazuje histogram czasu przyjścia impulsu przed korekcją, po korekcji względnej i po korekcji bezwzględnej dla trzech pikseli, a na rysunku \ref{fig:time_corr_hist_all} histogram czasu przyjścia dla wszystkich pikseli z wyłączeniem tych co miały mały sygnał, oraz wykluczam te dla których występowało duże odchylenie czasu przyjścia po dokonaniu korekcji, spowodowane, to jest problemami z kalibracją ponieważ dla tych pikseli podczas kalibracji impulsy wychodzi poza region zainteresowania. Wykluczyć musiałem 105 pikseli, czyi 1750 pikseli działo dobrze. \\
Rozkład nieskalibrowanych czasów przyjścia pokazuje wiele pików z powodu dyskretnych wartości, przyjmowanych przez impulsy podczas wyznaczania czasu przyjścia. Ta struktura  znika już po dokonaniu  względnej kalibracji czasu i odchylenie czasu przyjścia impulsu zmniejsza się z $\sim$1.7\,ns do $\sim$1\,ns, a po uwzględnieniu też średniego czasu przyjścia impulsu dla wszystkich pikseli w danym zdarzeniu, zmniejsza się ono do $\sim$0.4\,ns

\begin{figure}[H] 
\centering
\includegraphics[scale=0.25]{wykresy/drs4/time_corr_hist_3pixels.pdf}
\caption{Histogram czasu przyjścia dla 3 pikseli przed i po korekcjach czasowych.}
\label{fig:time_corr_hist3}
\end{figure}
\begin{figure}[H] 
\centering
\includegraphics[scale=0.3]{wykresy/drs4/time_corr_hist_all.pdf}
\caption{Histogram czasu przyjścia impulsu przed i po korekcjach czasowych dla wszystkich pikseli dla 1500 zdarzeń.}
\label{fig:time_corr_hist_all}
\end{figure}
W tabeli \ref{tab:std} przedstawiam odchylenie standardowe czasu przyjścia impulsu przed i po korekcjach dla trzech wybranych pikseli (dla których pokazałem histogramy czasu przyjścia przed i po korekcjach) oraz dla wszystkich działających pikseli. Na wykresie \ref{fig:std_hist_time_corr} przedstawiam histogram odchylenia standardowego przed i po dokonaniu korekcji dla wszystkich pikseli z wyłączeniem tych o których pisałem wcześniej. Na rysunku \ref{fig:std_map_time_corr} pokazuje mapę odchylenia standardowego dla każdego piksela. Czarne punkty są właśnie tymi pikselami, gdzie występował jakiś problem.
\begin{table}[H]
\caption{Odchylenie standardowe czasu przyjścia impulsu.}
\begin{tabular}{|l|c|c|c|l|}
\hline
Piksel id & Przed korekcją  & Po korekcji względnej  & po korekcji bezwzględnej   \\ \hline
180  & 1.70  & 1.03  & 0.42    \\ \hline
1440 & 1,19  & 1.01  & 0.42    \\  \hline
1590 & 2.05  & 1.02  & 0.44    \\  \hline
Wszystkie & 1.70  & 1.02 & 0.42   \\  \hline
\end{tabular}
\label{tab:std}
\end{table}

\begin{figure}[H] 
\centering
\includegraphics[scale=0.35]{wykresy/drs4/std_arrival_time_hist2.pdf}
\caption{Histogram odchylenia standardowego przed i po korekcji względnej oraz bezwzględnej.}
\label{fig:std_hist_time_corr}
\end{figure}

\begin{figure}[H] 
\centering
\includegraphics[scale=0.45]{wykresy/drs4/std_arrival_time_map2.pdf}
\caption{Mapa kamery z przedstawionym odchyleniem standardowym przed i po korekcji bezwzględnej wraz z przedstawionymi na czarno pikselami które zostały wykluczone z analizy z powodu problemów sprzętowych.}
\label{fig:std_map_time_corr}
\end{figure}
\newpage
\section{Zastosowanie korekcji do analizy obrazów pęków}
W tym rozdziale prezentuje zastosowanie korekcji opisanych w dwóch poprzednich rozdziałach do analizy obrazów pęków. Pokazuje jak korekcje linii bazowej oraz interpolacja szumów niegaussowskich obniża szum sygnału, w wyniku czego użyteczny sygnał jest dostawany, oraz jak korekcja czasowa pozwala lepiej wyznaczyć rozwój pęku w czasie.
\begin{figure}[H] 
\centering
\includegraphics[scale=0.3]{wykresy/drs4/shower_id_27.pdf}
\caption{Korekcja linii bazowej histogram, wszystkie piksele.}
\label{fig:muon_image}
\end{figure}

\begin{figure}[H] 
\centering
\includegraphics[scale=0.3]{wykresy/drs4/waveform_HG_id_27.pdf}
\caption{Korekcja linii bazowej histogram, wszystkie piksele.}
\label{fig:muon_image}
\end{figure}

\begin{figure}[H] 
\centering
\includegraphics[scale=0.3]{wykresy/drs4/waveform_LG_id_27.pdf}
\caption{Korekcja linii bazowej histogram, wszystkie piksele.}
\label{fig:muon_image}
\end{figure}

\begin{figure}[H] 
\centering
\includegraphics[scale=0.5]{wykresy/drs4/time_corr_image_id_27_clean.png}
\caption{Korekcja linii bazowej histogram, wszystkie piksele.}
\label{fig:muon_image}
\end{figure}

\begin{figure}[H] 
\centering
\includegraphics[scale=0.35]{wykresy/drs4/shower_id_85.pdf}
\caption{Korekcja linii bazowej histogram, wszystkie piksele.}
\label{fig:muon_image}
\end{figure}

\begin{figure}[H] 
\centering
\includegraphics[scale=0.3]{wykresy/drs4/waveform_HG_id_85.pdf}
\caption{Korekcja linii bazowej histogram, wszystkie piksele.}
\label{fig:muon_image}
\end{figure}

\begin{figure}[H] 
\centering
\includegraphics[scale=0.3]{wykresy/drs4/waveform_LG_id_85.pdf}
\caption{Korekcja linii bazowej histogram, wszystkie piksele.}
\label{fig:muon_image}
\end{figure}

\subsection{Poszukiwanie mionów}
\begin{figure}[H] 
\centering
\includegraphics[scale=0.3]{wykresy/drs4/shower_id_184.pdf}
\caption{Korekcja linii bazowej histogram, wszystkie piksele.}
\label{fig:muon_image}
\end{figure}

\begin{figure}[H] 
\centering
\includegraphics[scale=0.3]{wykresy/drs4/shower_id_12.pdf}
\caption{Korekcja linii bazowej histogram, wszystkie piksele.}
\label{fig:muon_image}
\end{figure}
\newpage
\section{Podsumowanie}





\begin{thebibliography}{9}
\bibitem{particle_de_angelis}
A. De Angelis, M. J. M. Pimenta.
\textit{Introduction to Particle and Astroparticle Physics} Springer 2015

\bibitem{auger_web}
\url{https://www.auger.org/}

\bibitem{telescope_array_web}
\url{http://www.telescopearray.org/}

\bibitem{Gamma-ray_article}
A. De Angelis, M. Mallamaci.
\textit{Gamma-Ray Astrophysics} 
\url{https://arxiv.org/abs/1805.05642}

\bibitem{TeVCat}
\url{tevcat.uchicago.edu}
Stan na 22.03.2019
\bibitem{IACT}
J. Holder
\textit{Atmospheric Cherenkov Gamma-ray Telescopes}
\url{https://arxiv.org/abs/1510.05675}

\bibitem{whipple}
T. C. Weekes et al.
\textit{ApJ. 342, 379–395, (1989).}

\bibitem{astro_particle}
Claus Grupen
\textit{Astroparticle Physics} Springer 2005

\bibitem{monte_carlo}
F. Schmidt, J. Knapp
\textit{"CORSIKA Shower Images", 2005}
\url{https://www-zeuthen.desy.de/~jknapp/fs/showerimages.html}

\bibitem{hilas}
A. Hillas, 
\textit{Cerenkov light images of EAS produced by primary gamma},
Proc. 19nd I.C.R.C. (La Jolla), Vol 3, 445 (1985)

\bibitem{cta_web}
\url{https://www.cta-observatory.org/}

\bibitem{dragon_lst}
S. Masuda, et al., 
\textit{Development of the photomultiplier tube readout
system for the first Large-Sized Telescope of the
Cherenkov Telescope Array}
\url{arXiv:1509.00548}

\bibitem{meg_experiment}
J. Adam, et al.,
\textit{The MEG detector for $\mu$ + $\rightarrow$ e + $\gamma$ decay search}, Euro. Phys. J. C 73 (2013), no. 4
\url{arXiv:1303.2348}

\bibitem{drs4_magic}
Sitarek, J., Gaug, M., Mazin, D., Paoletti, R., Tescaro, D., \textit{Analysis techniques and performance of the Domino Ring Sampler version 4 based readout for the MAGIC
telescopes}, Nuclear Instruments and Methods in Physics Research A, 723, 109, (2013)

\bibitem{drs4_psi}
DRS4 datasheet rev. 0.9
\url{https://www.psi.ch/sites/default/files/import/drs/DocumentationEN/DRS4_rev09.pdf}

\bibitem{lstchain_url}
\url{https://github.com/cta-observatory/cta-lstchain}

\bibitem{ctapipe_url}
\url{https://github.com/cta-observatory/ctapipe}
\end{thebibliography}
\end{document}